\cqusetup{
%	************	注意	************
%	* 1. \cqusetup{}中不能出现全空的行,如果需要全空行请在行首注释
%	* 2. 不需要的配置信息可以留空或删除
%	*
%	********************************
% ===================
%	论文的中英文题目
% ===================
  ctitle = {\ce{C60}表面的纳米形貌的可控性转变\\兼论\LaTeX{}在论文排版中的运用},
  etitle = {To Use \LaTeX{} in the Typeseting of\\Graduating Work for CQU},
% ===================
% 作者部分的信息
% ===================
  cauthor = 李振楠,
  eauthor = Zhennan~Li,
  studentid = 20128888,
  csupervisor = 孙麟\hskip\ccwd{}教授,
  esupervisor = Prof.~Lin Sun,
  edgree = {Degree of Master of Enginnering},
% 提示:如果内容太长,可以用\zihao{}命令控制字号,作用范围:{}内
  cmajor = {\zihao{-4}材料科学与工程(材料科学专业方向)},
  emajor = Material Science And Engineering,
% ===================
% 底部的学院名称和日期
% ===================
  cdepartment = 材料科学与工程学院,
  edepartment = College of Material Science And Engineering,
% ===================
% 封面的日期可以自动生成,也可以解除注释手动指定
% ===================
%	cdate = ,
%	edate = ,
% ===================
% 论文的关键词,使用英文逗号分隔
% ===================
  ckeywords = {重庆大学,\LaTeX,\LaTeXe,论文,模板},
  ekeywords = {bachelor, master, doctor, all support, white space is okay here}
	}% End of \cqusetup
% ===================
%
% 论文的摘要
%
% ===================
\begin{cabstract}
	\cquthesis{}是重庆大学毕业论文的\LaTeX{}模板,支持学士、硕士、博士论文的排版。合理使用本模板可以大大减轻重庆大学毕业生在毕业论文撰写过程中的排版工作量。
	
	\cquthesis{}根据重庆大学《重庆大学本科设计(论文)撰写规范化要求(2007年修订版)》和《重庆大学博士、硕士论文撰写格式标准(2007年修订版)》编写,力求合规,简洁,易于实现,用户友好。
	
	这个模板是站在巨人肩膀上的成果,感谢\LaTeXe{}计划,感谢薛瑞尼副教授(Github: xueruini/ThuThesis),感谢WeiJianWen同学(Github: weijianwen/SJTUThesis),感谢中国科学技术大学TeX用户组(Github: ustctug/gbt-7714-2015)。向你们致以真诚的问候和感激。
	
	本模板的特色:
  \begin{itemize}
  	\item 用例子来解释模板的使用方法;
  	\item 用废话来填充无关紧要的部分;
  	\item 一边学习摸索一边编写新代码。
  	\item 向薛教授学习 :-)
  \end{itemize}
\end{cabstract}
% 如果习惯关键字跟在摘要文字后面,可以用直接命令来设置,如下:
% \ckeywords{重庆大学,\LaTeX,\LaTeXe,论文,模板}

\begin{eabstract}
	LaTeX is a document preparation system for high-quality typesetting. It is most often used for medium-to-large technical or scientific documents but it can be used for almost any form of publishing.
	
  LaTeX contains features for:
\begin{enumerate}
  	\item Typesetting journal articles, technical reports, books, and slide presentations.
  	\item Control over large documents containing sectioning, cross-references, tables and figures.
  	\item Typesetting of complex mathematical formulas.
  	\item Advanced typesetting of mathematics with AMS-LaTeX.
  	\item Automatic generation of bibliographies and indexes.
  	\item Multi-lingual typesetting.
  	\item Inclusion of artwork, and process or spot colour.
  	\item Using PostScript or Metafont fonts.
  \end{enumerate}
  (Quote from \textit{https://latex-project.org/intro.html})  
\end{eabstract}

% \ekeywords{bachelor, master, doctor, all support, white space is okay here}